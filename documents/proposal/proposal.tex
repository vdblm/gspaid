\documentclass{article}
\usepackage{amsmath}
\usepackage{siunitx}
\usepackage{tikz}
\usepackage{graphicx}
\usepackage{mathpazo}
\usepackage[normalem]{ulem}
\usetikzlibrary{shapes, arrows}
\usepackage{hyperref}
\hypersetup{
    colorlinks=true,
    linkcolor=blue,
    filecolor=magenta,      
    urlcolor=cyan,
}
\usepackage[a4paper, margin=1.9cm]{geometry}
\usepackage{xepersian}



\title{به نام خدا\\پروپوزال سامانه پرداخت آنلاین ارزی}
\author{علی عسگری\\ وحید بالازاده \\ امین خدائی}

\begin{document}
\maketitle
\section{اهداف و توجیه}
\subsection{مقدمه}
به دلیل حساسیت‌های موجود نسبت به دانشجویان درس تحلیل و طراحی، دریافت ساده نمره از استاد و تی‌ای مقدور نیست و دریافت نمره‌ قبولی روندی دشوار است. از طرفی نیز برای بسیاری از امور من جمله ازدواج، پیدا کردن کار و یا درخواست پذیرش از دانشگاه‌های خارج از کشور نیاز به نمره‌ بالا است. با توجه به این که خرید مدرک مناسب کاری گران است، بعضی از اساتید به همراهی تی‌ای‌های محترم‌شان زحمت فراوان کشیده و به یاری دانشجویان آمده و با قراردادن پروژه‌‌هایی راهی برای گرفتن نمره فراهم کرده‌اند.
\subsection{معرفی و بیان اهداف پروژه}
\begin{itemize}
\item
تجربه انجام یک پروژه شبه واقعی
\item
تمرین کار تیمی در یک پروژه میان‌مدت
\item
دریافت نمره
\end{itemize}
\subsection{توجیه انجام پروژه}
علل بی‌توجهی به غایت پروژه:
\begin{itemize}
\item
کم‌ارزشی: ارزش چندانی تولید نمی‌کند.
\item
تکراری بودن و وجود رقبای جا افتاده
\item
بازار کوچک و درنتیجه کوچک بودن فضای اقتصادی این پروژه
\item
موقت بوده شیوه درآمدی
\item
پرهزینه بودن نیروی مورد نیاز برای کار موسسه به علت دستی بودن کار‌ها
\end{itemize}
علل موجه بودن انجام پروژه:
\begin{itemize}
\item
تصمیم تیم تدریس بر انجام این پروژه
\item
یادگیری
\item
دریافت نمره
\end{itemize}
امیدواریم با توجه به صداقت، درستی و توان‌مندی‌های اعضای این تیم که در قسمت معرفی تیم خواهد آمد، به تیم ما برای انجام این پروژه اعتماد کنید.
\section{معرفی و مفاد قرارداد}
از طرف تیم، تعهد داده میشود امکانات کارکردی و غیرکارکردی که به شرح زیر است، طبق کیفیت مطرح شده در قسمت «نحوه پوشش نیازمندی‌های کارکردی و غیرکارکردی»، پوشش داده شود. ضمنا از تکنولوژی‌های مطرح شده در قسمت «معرفی سیستم و توضیح تکنولوژی‌های مورد استفاده» استفاده خواهد شد و طبق زمانبندی و ددلاینهای داده شده، پروژه تحویل داده خواهد شد.\\
همچنین انتظار میرود روند نمره دهی به تقسیم بندی زیر باشد.
\subsection{امکانات کارکردی (70 درصد نمره)}
\begin{itemize}
\item 
امکانات عمومی (20 درصد)
\begin{itemize}
\item
صفحه ارتباط با ما
\item
صفحه معرفی موسسه
\item
مشاهده نرخ لحظه‌ای ارز
\item
صفحه قوانین و نرخ کارمزد‌ها
\item
امکان انجام تبدیلات آزمایشی بین ارزی
\item
سه نقش مشتری، کارمند و مدیر
\item
سه ارز معتبر ریال، دلار و یورو
\item
اعمال کارمزد درصدی مطابق با قوانین در تبادلات
\item
اعمال کف و سقف مبلغ در تمامی پرداخت‌ها و تراکنش‌ها طبق قوانین
\item
امکان ارسال نوتیفیکیشن برای اطلاع‌رسانی امکانات یا رویداد‌های قابل توجه
\end{itemize}
\item
تراکنش‌ (20 درصد)
\begin{itemize}
\item
امکان ثبت‌نام در آزمون‌های بین‌المللی TOEFL, IELTS, GRE
\item
امکان پرداخت شهریه یا هزینه‌ فرم درخواست پذیرش دانشگاه‌های خارجی
\item
امکان پرداخت ارزی به شماره حساب خارج از کشور
\item
امکان برداشت ریالی از کیف پول و انتقال به حساب بانکی داخلی
\item
امکان پرداخت ناشناس به شماره حساب داخلی موجود
\item
امکان پرداخت ناشناس به شماره حساب داخلی ناموجود پس از ساخته‌شدن خود‌کار آن حساب
\item
شکست تراکنش در صورت عدم تایید کارمند

\end{itemize}
امکانات مشتری (20 درصد)
\begin{itemize}
\item
ثبت‌نام
\item
مشاهده و ویرایش اطلاعات مشتریی 
\item
مشاهده اعتبار هر ارز کیف پول
\item
افزایش اعتبار ریالی کیف پول
\item
تبدیل مقداری از اعتبار یک ارز به ارز دیگر
\item
مشاهده اطلاعات هر تبادل شامل، مقدار مصرفی از ارز اول، مقدار به دست‌آمده از ارز دوم و هزینه کارمزد قبل از تکمیل تبادل
\item
انجام تراکنش‌های تفریف شده به کمک اعتبار کیف پول
\item
مشاهده تاریخچه تراکنش‌ها
\item
مشاهده جرئیات کامل هر تراکنش
\item
تعیین روش اطلاع‌رسانی به مشتری
\item
اطلاع‌رسانی اطلاعیه‌ها و رسید تراکنش‌ها موفق و ناموفق از طریق انتخاب شده مشتری
\item
روش‌های اطلاع‌رسانی رایان‌نامه و پیامک
\end{itemize}
\item
امکانات کارمند (20درصد)
\begin{itemize}
\item
تکمیل تراکنش پس از تایید کارمند
\item
امکان مشاهده کلیه تراکنش‌ها
\item
امکان مشاهده جزئیات تراکنش‌ها
\end{itemize}
\item
امکانات مدیر (20 درصد)
\begin{itemize}
\item
مشاهده تمامی تراکنش‌ها
\item
مشاهده لیست مشتریان
\item
مشاهده لیست کارمندان
\item
مشاهده فعالیت‌های هر مشتری
\item
مشاهده فعالیت‌های هر مشتری
\item
مشاهده اطلاعات جزئی هر تراکنش
\item
قطع دسترسی یک مشتری یا کارمند
\item
افزایش موجودی حساب ریالی و ارزی موسسه
\item
مشاهده تاریخچه فعالیت‌ها خود
\item
ایجاد کارمند جدید
\item
مشاهده گزارش موحودی و گردش حساب ارزی
\item
تعیین میزان حقوق هر کارمند
\item
ارسال خودکار پیام خودکار به مدیر در صورت ناکافی بود بودن اعتبار ریالی
\item
امکان ایجاد تراکنش معتبر دلخواه

\end{itemize}
\end{itemize}
\subsection{  امکانات غیرکارکردی (30 درصد نمره)}
\begin{itemize}
\item
تست واحد و تست سیستمی برای 30 درصد از نیازمندی های کارکردی مطرح شده در قسمت قبل. (40 درصد)
\item
زمان بارگیری هر صفحه حداکثر یک ثانیه در صورتی که پایگاه داده کمتر از 1000 رکورد داشته باشد. (30 درصد)
\item
تضمین امنیت سایت با استفاده از سایتهای www.ponycheckup.com و www.ssllabs.com به طوری که 80 درصد موارد این تست توسط سایت ما اخذ شود. (30 درصد)
\end{itemize}
\section{معرفی سیستم و توضیح تکنولوژی‌های مورد استفاده}
\subsection{راهکارهای فنی پیشنهادی و ارزیابی راهکارها}
رجوع شود به جدول.
\begin{table}[h!]
\centering
\resizebox{\textwidth}{!}{
\begin{tabular}{|c|c|c|c|c|}
\hline
\textbf{جنبه} & \textbf{راهکار اول} & \textbf{راهکار دوم} & \textbf{راهکار سوم} & \textbf{راهکار چهارم} \\
\hline
\textbf{بستر} &
برنامه دسکتاپ مبتنی بر شبکه & برنامه دسکتاپ مبتنی بر شبکه & برنامه وب & برنامه وب \\
\hline
\textbf{فریمورک} &
جاوا اف ایکس & سی‌‌شارپ دات نت & جنگو & لاراول \\
\hline
\textbf{پایگاه داده} &
اوراکل & اس کیو ال سرور & پوستگره & مای اس کیو ال \\
\hline

\textbf{دانش تیم و به تبع آن هزینه} &
کم & به ندرت & مناسب & هیچ \\
\hline
\textbf{قابل استفاده بودن در دستگاه‌های مختلف} &
نامناسب & نامناسب & مناسب & مناسب \\
\hline
\textbf{هزینه سخت‌افزار} &
کم & کم & مناسب & مناسب \\
\hline
\textbf{مستندسازی و به تبع آن قابل نگهداری بودن} &
مناسب & مناسب & مناسب & مناسب \\
\hline
\end{tabular}
}
\end{table}
\subsection{انتخاب بهترین گزینه}

با توجه به این که برای تیم ما دانش قبلی برای پیاده‌سازی سامانه بسیار مهم بود و با توجه به هزینه‌ها و موارد دیگر راهکار سوم انتخاب شد. در بخش‌های زیر به جزئیات راهکار مطرح شده اشاره میشود.\\
\textbf{معماری انتخاب شده برای طراحی نرم‌افزار به صورت زیر است}:\\
به منظور این که سامانه قابل انتقال \footnote{\lr{Portable}} و به راحتی قابل اعزام \footnote{\lr{Deployment}} باشد و عملکرد محیط توسعه و تست نرم افزار یکسان باشند، از تکنولوژی داکر استفاده میکنیم، به‌علاوه سامانه روی هر ماشینی که از داکر پشتایبانی کند، قابل راه‌اندازی خواهد بود.
به دلایلی که در بخش 2و2 بیان شد، برای توسعه سامانه از فریمورک جنگو\footnote{\lr{Django Framework}}، سرور گونیکورن\footnote{\lr{Gunicorn}}، پایگاه داده‌ی پُستگره \footnote{\lr{Postgres Database}} و وب‌سرور انجین‌ایکس \footnote{\lr{Nginx}} استفاده خواهد شد.\\
معماری فریمورک جنگو تقریبا مشابه معماری \lr{MVC} است و برای پیاده‌سازی برنامه کاربردی نیاز به طراحی تعدادی مدل وجود دارد. همچنین با استفاده از کد پایتون، جریان داده از کاربر به سیستم و بالعکس توصیف شده و در نهایت نحوه نمایش داده‌ها به کاربر طراحی می‌شود. این فریمورک به برنامه‌نویس اجازه می‌دهد تا بتواند از پلتفرم‌های مختلف رابط کاربری استفاده کند. با توجه به بخش از فریمورک سمنتیک یو آی \footnote{\lr{Semantic UI}} برای پیاده‌سازی رابط کاربری تعاملی استفاده‌ خواهد شد.\\
برای پرهیز از طراحی یکپارچه و چالش‌های آن و ایجاد امکان مهاجرت به معماری میکروسرویس \footnote{\lr{Microservice Architecture}} سامانه به چند زیرسامانه تقسیم شد که توضیح مختصری برای هرکدام در زیر آمده است:
\begin{itemize}
\item
\lr{Authorization}: در این زیرسامانه اطلاعات کاربران و دسترسی‌های آنان نگهداری و پردازش می‌شود،
\item
\lr{Management}:‌در این زیرسامانه قابلیت‌ها و امکانات مدیریتی سامانه برای مدیر آن پیاده‌سازی‌ می‌شود،
\item
\lr{Finance}: در این زیرسامانه اطلاعات حساب‌های مالی و تراکنش‌ها نگهداری و پردازش می‌شوند،
\item
\lr{Workflow}:‌در این سامانه درخواست‌های کاربران و روند کارشان از دریافت درخواست تا انجام یا تغییر وضعیت آن توسط کارمندان به انجام می‌رسد. 
\end{itemize}
در شکل زیر، خلاصه معماری موردنظر آمده است:\\
\textbf{توصیف جریان داده به شرح زیر است}:\\
همان‌ طور که بیان شد سامانه بر پایه وب خواهد بود و کاربران با استفاده از مرورگر به سامانه متصل می‌شوند و درخواست‌‌ها و کارهایشان را به انجام می‌رسانند. در اینجا مقصود از کاربران همه افراد استفاده‌کننده از سیستم اعم از مدیر، کارمندان و مشتری می‌باشد.\\
برای همه انواع کاربران ابتدا صفحه اولیه سایت نشان داده‌ می‌شود و کاربران ابتدا از طریق گزینه ثبت نام، در سایت ثبت نام کرده و سپس با استفاده از قسمت ورود، وارد محیط کاربری خود می‌شوند. سپس  هر کدام از درخواستها و یا کارهایی را که به آن دسترس دارند را انجام می‌دهند که شامل جست‌و‌جو در سایت با کلیک روی لینک‌ها و یا دادن اطلاعات به سایت از طریق فرم‌هاست. همچنین نتیجه عملیات کاربران در پی عملیاتشان به آن‌ها نمایش داده‌ می‌شود. از ابتدای یک درخواست مرورگر کاربر بر بستر HTTP درخواستی را شکل می‌دهد و به سامانه ارسال می‌کند. سپس وب‌سرور انجین‌ایکس که در نقش پروکسی قرار دارد در خواست را دریافت کرده و به نحو مناسبی در اختیار سامانه  قرار می‌دهد. این درخواست ابتدا به انجین‌ایکس می‌رسد که عملیات رمزنگاری و توزیع بار را انجام می‌دهد، سپس به لایه‌ی {\lr{WSGI} می‌رسد و کد پایتونیِ سایت توسط گونیکورن اجرا می‌شود. درخواست ابتدا از میان‌ابزار \footnote{\lr{Middleware}} های جنگو عبور می‌کند و با توجه به تنظیمات روتینگی که برای یو‌آرال‌های سایت انجام خواهد شد، یکی از توابع پایتونی‌، که پیاده‌سازی خواهیم کرد اجرا می‌شود و به درخواست، پاسخ مناسب می‌دهد. در این تابع در صورت لزوم ارتباط با پایگاه داده از طریق مدل‌ رابطه‌ای اشیای جنگو انجام می‌شود و منطق‌های مورد نیاز که در مدل‌های سایت پیاده‌سازی شده‌اند فراخوانی می‌شوند. در این صورت یک یا چند درخواست به پایگاه داده ‌هم ارسال خواهدشد. \\
برای تولید پاسخ درخواست، از امکان جنگو برای نمایش داده‌ها در قالب‌های رابط کاربری استفاده می‌شود. پس از این که پاسخ درخواست وب در تابع فراخوانی‌شده آماده و بازگردانده شد، جنگو پاسخ را از طریق گونیکورن در اختیار وب سرور قرار داده و وب‌سرور هم پاسخ کاربر را ارسال می‌کند و کاربر قادر خواهد بود نتیجه کار را در مرورگر خود مشاهده کند. ضمنا باید گفت هر کدام از سرویس‌های گونیکورن، پایگاه داده و انجین‌ایکس در کانتینرهای داکر اجرا‌ خواهند شد و ارتباط بینشان به طرزی نامرئی توسط شبکه داخل سرور انجام می‌شود.


\section{نحوه پوشش نیازمندی‌های کارکردی و غیرکارکردی}
\subsection* {پوشش نیازمندی‌های کارکردی}
‌برای تضمین پوشش نیازمندی‌ها، که در قسمت معرفی و مفاد قرارداد به تفضیل شرح داده شده است، ما از تست واحد و تست سیستمی استفاده می‌کنیم. خوشبختانه ‌فریمورک جنگو از هر دوی این نوع تست‌ها پشتیبانی می‌کند و ما تضمین می کنیم که برای 30 درصد از موارد نیازمندی‌ها که همان نیازمندی‌هایی هستند که به نظر و صلاحدید ما، مهم‌تر یا حساس‌تر هستند، تست مستقیم نوشته‌شده باشد.

\subsection* {پوشش نیازمندی‌های غیر کارکردی}
از نظر سرعت سامانه زیاد حساس نیست و ما تضمین می‌کنیم مادامی که پایگاه داده کمتر از 1000 رکورد داشته‌باشد، کاربران بدون محدودیت شبکه قادر خواهند بود هر صفحه را در کمتر از یک ثانیه بارگیری کنند. 

برای تضمین امنیت سایت ما از تست سایت‌های www.ponycheckup.com و www.ssllabs.com استفاده خواهیم کردیم  و این که 80 درصد موارد این تست‌ها توسط سایت ما اخذ شود، تضمین می‌شود. \\
قابل انتقال و اعزام بودن نرم افزار با توجه به تکنولوژی‌ای که استفاده کرده‌ایم تضمین شده‌است و کافی است رایانه مقصد از تکنولوژی داکر ورژن پیاده‌سازی ما پشتیبانی کند. 

\newpage
\section{مراحل اجرای فازها و برآورد زمانی و مالی کلی}
‌\subsection* {تخمین زمانی و هزینه پرسنل}
با توجه به بازار موجود و پرس و جوی میدانی به دانشجویان کارشناسی که وب توسعه می‌دهند ساعتی 20 هزار تومان دستمزد تعلق می‌گیرد. همچنین هزینه‌ها بر اساس ریال بیان شده‌است. همچنین برای تخمین زمان پیاده‌سازی از ضریب اطمینان 4 استفاده شده‌است.
\begin{table}[h!]
\centering
\resizebox{\textwidth}{!}{
\begin{tabular}{|c|c|c|c|}
\hline
\textbf{بخش} & \textbf{نفرساعت} & \textbf{نرخ} & \textbf{مجموع} \\
\hline
\textbf{تهیه پروپوزال} &
48 & 200000 & 9600000 \\
\hline
\textbf{تهیه WBS} &
32 & 200000 & 6400000 \\
\hline
\textbf{طراحی سیستم از جمله موردهای کاربردی یا یوزکیس} &
64 & 200000 & 12800000 \\
\hline
\textbf{پیاده‌سازی و اعزام و تحویل پروژه} &
1600 & 200000 & 320000000 \\
\hline
\textbf{مجموع} &
- & - & 348800000 \\
\hline
\end{tabular}
}
\end{table}
\section{معرفی اعضا و تقسیم کار در گروه}
\subsection{معرفی تیم}
علی عسگری خشویه\\
دانشجوی کارشناسی مهندسی کامپیوتر(نرم افزار)\\
دانشگاه صنعتی شریف\\
تلفن همراه: 09136496628 \\
رایانامه: \lr{aliasgari@ce.sharif.edu}\\
\\
سید محمدامین خدائی\\
دانشجوی کارشناسی مهندسی کامپیوتر(نرم افزار)\\
دانشگاه صنعتی شریف\\
تلفن همراه: 09374748373\\
رایانامه: \lr{khodmamin@gmail.com}\\
\\
وحید بالازاده مرشت\\
دانشجوی کارشناسی مهندسی کامپیوتر(نرم افزار)\\
دانشگاه صنعتی شریف\\
تلفن همراه: 09381137897\\
رایانامه: \lr{balazadehvahid@gmail.com}
\subsection{تقسیم وظایف}
تیم فنی، که از هر سه عضو تشکیل شده و با مدیریت فنی آقای علی عسگری فعالیت می‌کند، مسئولیت پیاده‌سازی قسمت‌های زیرساخت، فرانت‌اند، بک‌اند و تضمین کیفیت و امینت را بر عهده دارد.
\subsection{بیان نمرات اضافه}
قابل انتقال و اعزام بودن نرم افزار با توجه به تکنولوژی‌ای که استفاده کرده‌ایم تضمین شده‌است و کافی است رایانه مقصد از تکنولوژی داکر ورژن پیاده‌سازی ما پشتیبانی کند. برای این قسمت انتظار میرود نمره اضافه به اندازه 20 درصد از نمره کل پروژه در نظر گرفته شود.

\end{document}
