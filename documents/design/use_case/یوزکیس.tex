%hello
%please do the todos
\documentclass[11pt]{article}
%\usepackage{changepage}
%\usepackage{lipsum} % just for the example
\usepackage{amsmath}
\usepackage{siunitx}
\usepackage{tikz}
\usepackage{graphicx}
\usepackage{mathpazo}
\usepackage[normalem]{ulem}
\usetikzlibrary{shapes, arrows}
\usepackage{hyperref}
\hypersetup{
    colorlinks=true,
    linkcolor=blue,
    filecolor=magenta,      
    urlcolor=cyan,
}
\usepackage{amsmath}
\usepackage{xepersian}
\usepackage{graphicx}
\settextfont{XB Niloofar}
\setlatintextfont{Linux Libertine O}
\setdigitfont{XB Niloofar}
\title{
موارد کاربردی جی‌اس‌پید
}
\author{سید محمد‌امین خدائی(۹۴۱۰۹۳۱۳)}


\begin{document}
\maketitle
\section{سوال ۱}
\subsection{الف}
صحیح: 
\begin{itemize}
\item
معمولا فاصله‌ بین دو بار تکرار یک کلمه عدد بزرگی نیست، و اگر این اعداد را در حافظه‌های بزرگی ذخیره کنیم،‌ مقدار بیشتری از حافظه بیهوده می‌ماند.
\item
پردازش‌گر‌ها برای فعالیت‌با حافظه‌های بزرگ‌تر (wordها) طراحی‌شده و بهینه‌سازی شده‌اند. لذا حجم حافظه بیشتر سرعت پردازش را بالا می‌برد.
\end{itemize}
\subsection{ب}
غلط: هنگامی که واحد سند بزرگ‌تر می‌شود، برای مثال دو قسمت A و B را کنار هم به عنوان یک سند می‌بینیم: اگر a در A است و b در B، و وقتی کوئری ab می‌آید، گاهی این به معنی این‌است که سند AB مربوط به این کوئری هست و این recall را بالا می‌برد و گاه این به این معنی‌ست که AB مربوط به این کوئری نیست، و این accuracy را پایین می‌آورد.
\subsection{ج}
صحیح: در صورتی که مربوط بودن را مربوط بودن واقعی در نظر بگیریم، عمل‌وند and جواب‌ها را به شدت محدود کرده، و جواب‌هایی که برمی‌گرداند ارتباط بیشتری با کوئری دارند (افزایش accuracy) چون هر دو سری کلمه سمت چپ و راست and را حتما دارند. و از طرف دیگر مستنداتی که هر دو را ندارند ولی مثلا یکی را خیلی زیاد دارند و دیگری را با عبارات دیگری مطرح کرده‌اند بازیابی نکرده و recall را کاهش می‌دهد.
\subsection{د}
غلط: بر اساس اسلاید ۷م مجموعه اسلاید ۶م، 	حذف stopword‌ها در این مجموعه داده بسیار موثر‌تر از stemming بوده است. به نظرم دلیل آن بسیار پر تکرار بودن stopwordها است.

\section{سوال ۲}
\subsection{الف}
به طور کلی هر چه طول گام‌های skippointer را بیشتر کنیم، احتمال اینکه این گام‌ها استفاده شوند کم‌تر می‌شود، ولی در صورت استفاده میزان سود ما بیشتر می‌شود. 
\subsection{ب}
۱ -> ۴ -> ۶ -> ۱۰ -> ۱۲ -> ۲۰ -> ۲۲ -> ۳۲ -> ۴۷ -> ۸۱ 
\newline
۳ -> ۱۰ -> ۳۲ ->‌ ۳۵ -> ۴۷ -> ۵۱ -> ۵۶ -> ۸۰
\newline
تعداد اعدادی که دیده شد: ۱۸
\subsection{ج}
حیر، زیرا وقتی تفاوتی در دو لیست می‌بینیم نباید آن را skip کنیم بلکه باید آن را جزو جواب بیاوریم و به همین دلیل پردازش همه داده‌ها لازم است.
\subsection{د}
می‌توانیم از پایان عدد k ام به شروع عدد
\lr{k + i} 
ام اشاره‌گری بگذاریم و کنار اشاره‌گر مجموع تفاوت‌ها عدد
\lr{k + i} 
ام با عدد kام را بنویسیم.
\section{سوال ۴}
$$ zarib = \frac{1}{1.61 - 1.25} = 2.8 $$
\subsection{الف}
$$ 25 \times 50000 = 1250000 Bytes = 1.25 GBytes$$
\subsection{ب}
$$ \frac{50000}{i^2} \times 2.8 = {140000}{i^2} $$
\subsection{ج}
$$ \frac{140000}{i^2} \times i = \frac{140000}{i}$$
\subsection{د}
$$ stringlen = 140000 \times \sum \frac{1}{i} = 140000 \times 2.27 = 317800$$
$$ pointerlen = 50000 \times [\frac{\log _2 stringlen}{8}] = 50000 \times 3 = 150000 $$
$$ memory = stringlen + pointerlen = 467800 Bytes = 467.8 MBytes $$
\subsection{ه}
$$ stringlen = D:stringlen + 50000 = 317800 + 50000 = 367800 $$
$$ pointerlen = [\frac{50000}{8}] \times [\frac{\log _2 stringlen}{8}] = 6250 \times 3 = 18750 $$
$$ memory = stringlen + pointerlen = 386550 Bytes = 386.55 MBytes $$

\section{سوال ۵}
\subsection{الف}
دنباله فاصله‌ها به صورت بیتی:
$$ 1001, 110, 11, 111011, 111 $$
دنباله فاصله‌ها:
$$ 9, 6, 3, 59, 7 $$
دنباله‌ DocID ها: 
$$ 9, 15, 18, 77, 84 $$
\subsection{ب}
$$ 0001001100001101000001110111011100001111 $$

\section{سوال ۷}
به طور کلی خیر. زیرا با زیاد کردن تعداد داکیومنتی که بازمی‌گردانیم، همواره recall را ثابت‌نگه داشته یا بیشتر می‌کنیم و precision را ثابت نگه می‌داریم و کاهش می‌دهیم. تنها حالتی که ممکن از در دو نقطه با هم برابر باشند این است که در بازه‌ای که با هم برابرند شرایط به گونه‌ای باشد که precision و recall در حالت ثبات خود باشند، ولی این قسمت اگر وجود داشته باشد هم قسمتی پیوسته است.
\section{سوال ۸}
\subsection{الف}
$$ precision = \frac{6}{20} = 0.30 $$
$$ recall = \frac{6}{8} = 0.75 $$
\subsection{ب}
$$ F = \frac{2PR}{P + R} = 0.43 $$
\subsection{ج}
$$ MAP = \frac{\frac{1}{1} + \frac{2}{6} + \frac{3}{10} + \frac{4}{12} + \frac{5}{19} + \frac{6}{20}}{6} = 0.42 $$
\subsection{د}
\begin{enumerate}
\item
$$ MAP = \frac{\frac{1}{1} + \frac{2}{6} + \frac{3}{10} + \frac{4}{12} + \frac{5}{19} + \frac{6}{20} + \frac{7}{21} + \frac{8}{22}}{8} = 0.4 $$
\item
$$ MAP = \frac{\frac{1}{1} + \frac{2}{6} + \frac{3}{10} + \frac{4}{12} + \frac{5}{19} + \frac{6}{20} + \frac{7}{9999} + \frac{8}{10000}}{8} = 0.32 $$


\end{enumerate}
\end{document}