\documentclass{article}
\usepackage{amsmath}
\usepackage{siunitx}
\usepackage{tikz}
\usepackage{graphicx}
\usepackage{mathpazo}
\usepackage[normalem]{ulem}
\usetikzlibrary{shapes, arrows}
\usepackage{hyperref}
\hypersetup{
    colorlinks=true,
    linkcolor=blue,
    filecolor=magenta,      
    urlcolor=cyan,
}
\usepackage[a4paper, margin=1.9cm]{geometry}
\usepackage{xepersian}


\title{موارد کاربردی سامانه}
\settextfont{XB Zar}

\begin{document}
\maketitle
\begin{itemize}
\item \textbf{مورد کاربرد:}\\
مشاهده ی صفحه معرفی موسسه
\item \textbf{شناسه:}\\
1
\item \textbf{توضیح اجمالی:}\\
صفحه معرفی موسسه به کاربر نمایش داده میشود. 
\item \textbf{کنشگر اصلی:}\\
مشتری
\item \textbf{کنشگر فرعی:}\\
ندارد
\item \textbf{شرایط اولیه:}\\
ندارد
\item \textbf{روند اصلی:}\\
\begin{enumerate}
\item  این مورد کاربرد با درخواست کاربر برای مشاهدهی صفحه معرفی موسسه به کاربر شروع میشود
\item سامانه صفحه معرفی موسسه را به کاربر نمایش میدهد
\end{enumerate}
\item \textbf{شرایط پایانی:}\\ 
ندارد
\item \textbf{روندهای جایگزین:}\\
ندارد
\end{itemize}
\noindent\makebox[\linewidth]{\rule{\paperwidth}{0.4pt}}

\begin{itemize}
\item \textbf{مورد کاربرد:}\\
مشاهده نرخ لحظه ای ارز 
\item \textbf{شناسه:}\\
2
\item \textbf{توضیح اجمالی:}\\

در صفحه اولیه (خانه) کاربر نرخ لحظه ای ارزها را مشاهده میکند. 
\item \textbf{کنشگر اصلی:}\\
مشتری
\item \textbf{کنشگر فرعی:}\\
ندارد
\item \textbf{شرایط اولیه:}\\
ندارد
\item \textbf{روند اصلی:}\\
\begin{enumerate}
\item این مورد کاربرد با ورود کاربر به صفحه خانه شروع میشود
\item سامانه نرخ لحظه ای ارزها (دلار و یورو) را بر حسب ریال نشان میدهد. 
\end{enumerate}
\item \textbf{شرایط پایانی:}\\ 
ندارد
\item \textbf{روندهای جایگزین:}\\
ندارد
\end{itemize}
\noindent\makebox[\linewidth]{\rule{\paperwidth}{0.4pt}}
\begin{itemize}
\item \textbf{مورد کاربرد:}\\
مشاهده صفحه قوانین و نرخ کارمزدها 
\item \textbf{شناسه:}\\
۳
\item \textbf{توضیح اجمالی:}\\
صفحه قوانین و نرخ کارمزدها به کاربر نشان داده میشود. 
\item \textbf{کنشگر اصلی:}\\
مشتری
\item \textbf{کنشگر فرعی:}\\
ندارد
\item \textbf{شرایط اولیه:}\\
ندارد
\item \textbf{روند اصلی:}\\
\begin{enumerate}
\item  این مورد کاربرد با درخواست کاربر برای مشاهده صفحه قوانین و نرخ کارمزدها به کاربر شروع میشود.
\item سامانه قوانین و نرخ کارمزدها را به کاربر نشان میدهد.
\end{enumerate}
\item \textbf{شرایط پایانی:}\\ 
ندارد
\item \textbf{روندهای جایگزین:}\\
ندارد
\end{itemize}
\noindent\makebox[\linewidth]{\rule{\paperwidth}{0.4pt}}
\begin{itemize}
\item \textbf{مورد کاربرد:}\\
انجام تبدیلات آزمایشی بین ارزی 
\item \textbf{شناسه:}\\
4
\item \textbf{توضیح اجمالی:}\\
در صفحه اولیه (خانه) کاربر میتواند نتیجه تبدیل ارزها را با توجه به نرخ ارز و کارمزد به طور آزمایشی را مشاهده میکند. 
\item \textbf{کنشگر اصلی:}\\
مشتری
\item \textbf{کنشگر فرعی:}\\
ندارد
\item \textbf{شرایط اولیه:}\\
ندارد
\item \textbf{روند اصلی:}\\
\begin{enumerate}
\item این مورد کاربرد با ورود کاربر به صفحه خانه شروع میشود.
\item مشتری ارز مبدا، مقصد و مقدار دلخواه برای تبدیل را وارد میکند
\item در صورتی که مقدار ورودی صحیح باشد:
\begin{enumerate}
\item سامانه با توجه به نرخ لحظه ای ارز و مقدار کارمزد مقدار تبدیل شده صحیح را با توجه به ارز مقصد به مشتری نشان میدهد.  
\end{enumerate}
\item در غیر اینصورت:
\begin{enumerate}
\item سامانه پیغام خطایی مبنی بر نادرست بودن مقدار ورودی به مشتری میدهد
\end{enumerate}
\end{enumerate}
\item \textbf{شرایط پایانی:}\\ 
ندارد
\item \textbf{روندهای جایگزین:}\\
ندارد
\end{itemize}

\noindent\makebox[\linewidth]{\rule{\paperwidth}{0.4pt}}
\begin{itemize}
\item \textbf{مورد کاربرد:}\\
مشاهده ی صفحه معرفی موسسه
\item \textbf{شناسه:}\\
1
\item \textbf{توضیح اجمالی:}\\
صفحه معرفی موسسه به کاربر نمایش داده میشود. 
\item \textbf{کنشگر اصلی:}\\
مشتری
\item \textbf{کنشگر فرعی:}\\
ندارد
\item \textbf{شرایط اولیه:}\\
ندارد
\item \textbf{روند اصلی:}\\
\begin{enumerate}
\item  این مورد کاربرد با درخواست کاربر برای مشاهدهی صفحه معرفی موسسه به کاربر شروع میشود
\item سامانه صفحه معرفی موسسه را به کاربر نمایش میدهد
\end{enumerate}
\item \textbf{شرایط پایانی:}\\ 
ندارد
\item \textbf{روندهای جایگزین:}\\
ندارد
\end{itemize}
\noindent\makebox[\linewidth]{\rule{\paperwidth}{0.4pt}}
\end{document}
