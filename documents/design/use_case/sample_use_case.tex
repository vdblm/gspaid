\documentclass{article}
\usepackage{amsmath}
\usepackage{siunitx}
\usepackage{tikz}
\usepackage{graphicx}
\usepackage{mathpazo}
\usepackage[normalem]{ulem}
\usetikzlibrary{shapes, arrows}
\usepackage{hyperref}
\hypersetup{
    colorlinks=true,
    linkcolor=blue,
    filecolor=magenta,      
    urlcolor=cyan,
}
\usepackage[a4paper, margin=1.9cm]{geometry}
\usepackage{xepersian}


\title{موارد کاربردی سامانه}
\settextfont{XB Zar}

\begin{document}
\maketitle
\begin{itemize}
\item \textbf{مورد کاربرد:}\\
مشاهده ی صفحه معرفی موسسه
\item \textbf{شناسه:}\\
1
\item \textbf{توضیح اجمالی:}\\
صفحه معرفی موسسه به کاربر نمایش داده میشود. 
\item \textbf{کنشگر اصلی:}\\
مشتری
\item \textbf{کنشگر فرعی:}\\
ندارد
\item \textbf{شرایط اولیه:}\\
ندارد
\item \textbf{روند اصلی:}\\
\begin{enumerate}
\item  این مورد کاربرد با درخواست کاربر برای مشاهدهی صفحه معرفی موسسه به کاربر شروع میشود
\item سامانه صفحه معرفی موسسه را به کاربر نمایش میدهد
\end{enumerate}
\item \textbf{شرایط پایانی:}\\ 
ندارد
\item \textbf{روندهای جایگزین:}\\
ندارد
\end{itemize}
\noindent\makebox[\linewidth]{\rule{\paperwidth}{0.4pt}}

\begin{itemize}
\item \textbf{مورد کاربرد:}\\
مشاهده نرخ لحظه ای ارز 
\item \textbf{شناسه:}\\
2
\item \textbf{توضیح اجمالی:}\\

در صفحه اولیه (خانه) کاربر نرخ لحظه ای ارزها را مشاهده میکند. 
\item \textbf{کنشگر اصلی:}\\
مشتری
\item \textbf{کنشگر فرعی:}\\
ندارد
\item \textbf{شرایط اولیه:}\\
ندارد
\item \textbf{روند اصلی:}\\
\begin{enumerate}
\item این مورد کاربرد با ورود کاربر به صفحه خانه شروع میشود
\item سامانه نرخ لحظه ای ارزها (دلار و یورو) را بر حسب ریال نشان میدهد. 
\end{enumerate}
\item \textbf{شرایط پایانی:}\\ 
ندارد
\item \textbf{روندهای جایگزین:}\\
ندارد
\end{itemize}
\noindent\makebox[\linewidth]{\rule{\paperwidth}{0.4pt}}
\begin{itemize}
\item \textbf{مورد کاربرد:}\\
مشاهده صفحه قوانین و نرخ کارمزدها 
\item \textbf{شناسه:}\\
۳
\item \textbf{توضیح اجمالی:}\\
صفحه قوانین و نرخ کارمزدها به کاربر نشان داده میشود. 
\item \textbf{کنشگر اصلی:}\\
مشتری
\item \textbf{کنشگر فرعی:}\\
ندارد
\item \textbf{شرایط اولیه:}\\
ندارد
\item \textbf{روند اصلی:}\\
\begin{enumerate}
\item  این مورد کاربرد با درخواست کاربر برای مشاهده صفحه قوانین و نرخ کارمزدها به کاربر شروع میشود.
\item سامانه قوانین و نرخ کارمزدها را به کاربر نشان میدهد.
\end{enumerate}
\item \textbf{شرایط پایانی:}\\ 
ندارد
\item \textbf{روندهای جایگزین:}\\
ندارد
\end{itemize}
\noindent\makebox[\linewidth]{\rule{\paperwidth}{0.4pt}}
\begin{itemize}
\item \textbf{مورد کاربرد:}\\
انجام تبدیلات آزمایشی بین ارزی 
\item \textbf{شناسه:}\\
4
\item \textbf{توضیح اجمالی:}\\
در صفحه اولیه (خانه) کاربر میتواند نتیجه تبدیل ارزها را با توجه به نرخ ارز و کارمزد به طور آزمایشی را مشاهده میکند. 
\item \textbf{کنشگر اصلی:}\\
مشتری
\item \textbf{کنشگر فرعی:}\\
ندارد
\item \textbf{شرایط اولیه:}\\
ندارد
\item \textbf{روند اصلی:}\\
\begin{enumerate}
\item این مورد کاربرد با ورود کاربر به صفحه خانه شروع میشود.
\item مشتری ارز مبدا، مقصد و مقدار دلخواه برای تبدیل را وارد میکند
\item در صورتی که مقدار ورودی صحیح باشد:
\begin{enumerate}
\item سامانه با توجه به نرخ لحظه ای ارز و مقدار کارمزد مقدار تبدیل شده صحیح را با توجه به ارز مقصد به مشتری نشان میدهد.  
\end{enumerate}
\item در غیر اینصورت:
\begin{enumerate}
\item سامانه پیغام خطایی مبنی بر نادرست بودن مقدار ورودی به مشتری میدهد
\end{enumerate}
\end{enumerate}
\item \textbf{شرایط پایانی:}\\ 
ندارد
\item \textbf{روندهای جایگزین:}\\
ندارد
\end{itemize}

\noindent\makebox[\linewidth]{\rule{\paperwidth}{0.4pt}}
\begin{itemize}
\item \textbf{مورد کاربرد:}\\
تغییر و بروزرسانی صفحه قوانین
\item \textbf{شناسه:}\\
5
\item \textbf{توضیح اجمالی:}\\
صفحه قوانین بروز میشود.
\item \textbf{کنشگر اصلی:}\\
مدیر
\item \textbf{کنشگر فرعی:}\\
ندارد
\item \textbf{شرایط اولیه:}\\
مدیر وارد داشبورد ادمین شده باشد.
\item \textbf{روند اصلی:}\\
\begin{enumerate}
\item  این مورد کاربرد با درخواست مدیر برای ایجاد تغییر در صفحه قوانین شروع میشود.
\item سامانه صفحه  ایجاد تغییر قوانین را به مدیر نشان میدهد.
\item مدیر کف و سقف مبلغ تراکنشها و توضیحات مربوط به قوانین سایت را تغییر میدهد.
\item پیغام بروزرسانی موفق قوانین به مدیر نشان داده میشود.
\end{enumerate}
\item \textbf{شرایط پایانی:}\\ 
قوانین یا سقف و کف مبلغ تراکنشها تغییر میکند و صفحه قوانین بروز میشود.
\item \textbf{روندهای جایگزین:}\\
ندارد
\end{itemize}

\noindent\makebox[\linewidth]{\rule{\paperwidth}{0.4pt}}

\begin{itemize}
\item \textbf{مورد کاربرد:}\\
ارسال نوتفیکیشن مدیر
\item \textbf{شناسه:}\\
6
\item \textbf{توضیح اجمالی:}\\
برای اطلاع رسانی مشتری ها از امکانات جدید نوتیفیکیشن با استفاده از رایانامه یا پیامک ارسال میشود.
\item \textbf{کنشگر اصلی:}\\
مدیر
\item \textbf{کنشگر فرعی:}\\
ندارد
\item \textbf{شرایط اولیه:}\\
مدیر وارد داشبورد مدیر شده باشد.
\item \textbf{روند اصلی:}\\
\begin{enumerate}
\item مورد کاربردی با درخواست مدیر در «ایجاد تراکنش معتبر دلخواه» شروع میشود: 
\item پیغام نوتیفیکیشن برابر با پیغام وارد شده توسط مدیر میشود.
\item تمام مشتری ها به عنوان مشتری مقصد انتخاب میشوند.
\item با توجه به شیوه اطلاع رسانی ثبت شده توسط هر کدام از مشتری های مقصد پیغام مربوطه به آنها ارسال میشود.
\end{enumerate}

\item \textbf{شرایط پایانی:}\\ 
هر کدام از مشتریهای مقصد پیغام را از طریق روش اطلاع رسانی مشخص شده دریافت کند.
\item \textbf{روندهای جایگزین:}\\
ندارد
\end{itemize}
\noindent\makebox[\linewidth]{\rule{\paperwidth}{0.4pt}}

\begin{itemize}
\item \textbf{مورد کاربرد:}\\
ارسال نوتفیکیشن کارمند
\item \textbf{شناسه:}\\
7
\item \textbf{توضیح اجمالی:}\\
برای اطلاع رسانی مشتری ها از وضعیت درخواستشان نوتیفیکیشن با استفاده از رایانامه یا پیامک ارسال میشود.
\item \textbf{کنشگر اصلی:}\\
کارمند
\item \textbf{کنشگر فرعی:}\\
ندارد
\item \textbf{شرایط اولیه:}\\
کارمند وارد داشبورد کارمند شده باشد.
\item \textbf{روند اصلی:}\\
\begin{enumerate}
\item مورد کاربردی با درخواست کارمند در «بررسی تراکنش» یا «تایید تراکنش» شروع میشود
\item پیغام نوتیفیکیشن برابر با پیغام وارد شده توسط کارمند میشود.
\item مشتری مقصد برابر با مشتری در مورد کاربردی «بررسی تراکنش» یا «تایید تراکنش» میشود.
\item با توجه به شیوه اطلاع رسانی ثبت شده توسط مشتری مقصد پیغام مربوطه ارسال میشود.
\end{enumerate}

\item \textbf{شرایط پایانی:}\\ 
مشتری مقصد پیغام را از طریق روش اطلاع رسانی مشخص شده دریافت کند.
\item \textbf{روندهای جایگزین:}\\
ندارد
\end{itemize}
\noindent\makebox[\linewidth]{\rule{\paperwidth}{0.4pt}}

\begin{itemize}
\item \textbf{مورد کاربرد:}\\
ایجاد تراکنش معتبر دلخواه
\item \textbf{شناسه:}\\
8
\item \textbf{توضیح اجمالی:}\\
یک تراکنش دلخواه توسط مدیر تعریف میشود.
\item \textbf{کنشگر اصلی:}\\
مدیر
\item \textbf{کنشگر فرعی:}\\
ندارد
\item \textbf{شرایط اولیه:}\\
مدیر وارد داشبورد ادمین شده باشد.
\item \textbf{روند اصلی:}\\
\begin{enumerate}
\item  این مورد کاربرد با درخواست مدیر برای ایجاد یک تراکنش جدید شروع میشود.
\item سامانه صفحه ایجاد تراکنش جدید را به مدیر نشان میدهد.
\item مدیر عنوان تراکنش، مقدار هزینه آن، واحد پولی و کارمزد تراکنش و اطلاعات موردنیاز را وارد میکند.
\item در صورتی که این تراکنش از قبل در سامانه باشد:
\begin{enumerate}
\item سامانه پیغام خطا میدهد و به قسمت ۳ برمیگردد. 
\end{enumerate}

\item در غیراینصورت	:
\begin{enumerate}
\item پیغام ایجاد موفق تراکنش به مدیر نشان داده میشود.
\end{enumerate}
\item  درصورتی که مدیر گزینه ارسال نوتیفیکیشن را انتخاب کند:
\begin{enumerate}
\item مدیر پیام نوتیفیکیشن را وارد کرده و تایید میکند.
\item مورد کابردی «ارسال نوتیفیکیشن مدیر» انجام میشود.
\end{enumerate}
\item سامانه پیغام ایجاد موفق تراکنش را نشان میدهد.
\end{enumerate}

\item \textbf{شرایط پایانی:}\\ 
در قسمت تراکنشهای مربوط به کاربران، تراکنش ایجاد شده اضافه شود.
\item \textbf{روندهای جایگزین:}\\
ندارد
\end{itemize}


\noindent\makebox[\linewidth]{\rule{\paperwidth}{0.4pt}}

\begin{itemize}
\item \textbf{مورد کاربرد:}\\
ثبت نام در آزمون های بین المللی
\item \textbf{شناسه:}\\
9
\item \textbf{توضیح اجمالی:}\\
درخواست مشتری برای ثبت نام در آزمون انتخابی، در سامانه ثبت میشود.
\item \textbf{کنشگر اصلی:}\\
مشتری
\item \textbf{کنشگر فرعی:}\\
ندارد
\item \textbf{شرایط اولیه:}\\
مشتری وارد داشبورد شده باشد.
\item \textbf{روند اصلی:}\\
\begin{enumerate}
\item  این مورد کاربرد با درخواست مشتری برای ثبت درخواست جدید شروع میشود.
\item سامانه صفحه درخواست جدید را به مشتری نشان میدهد.
\item مشتری نوع آزمون مورد نظر را انتخاب میکند.
\item سامانه اطلاعات موردنیاز برای ثبت نام را به مشتری نمایش میدهد
\item مشتری اطلاعات موردنیاز را وارد میکند(یا به صورت متن یا به صورت فایل)
\item سامانه هزینه ثبت نام را به مشتری نمایش میدهد
\item مشتری درخواست خود را ثبت میکند
\item در صورتی که موجودی مشتری کمتر از هزینه ثبت نام باشد سامانه :
\begin{enumerate}
\item سامانه پیغام خطا میدهد و روند جایگزین ۱ انجام میشود. 
\end{enumerate}

\item در غیراینصورت	:
\begin{enumerate}
\item پیغام ثبت موفق تراکنش به مشتری نشان داده میشود.
\item آدرس مربوط به کارمند مسئول بررسی درخواست به مشتری نمایش داده میشود.
\item مورد کاربردی «بررسی تراکنش» اجرا میشود.
\end{enumerate}

\end{enumerate}

\item \textbf{شرایط پایانی:}\\ 
در قسمت تراکنشهای مربوط به کارمند مسئول، تراکنش ایجاد شده در حالت در انتظار تایید اضافه شود.
\item \textbf{روندهای جایگزین:}\\
روند ۱:\\
\begin{enumerate}
\item پیغام ورود به کیف پول نمایش داده میشود.
\item کاربر ورود به کیف پول را تایید میکند.
\item مورد کاربردی «شارژ کیف پول » اجرا میشود.؟؟
\end{enumerate}

\end{itemize}


\noindent\makebox[\linewidth]{\rule{\paperwidth}{0.4pt}}

\begin{itemize}
\item \textbf{مورد کاربرد:}\\
پرداخت شهریه یا هزینه فرم درخواست پذیرش دانشگاههای خارجی
\item \textbf{شناسه:}\\
10
\item \textbf{توضیح اجمالی:}\\
درخواست مشتری برای پراخت هزینه پذیرش دانشگاههای خارجی در سامانه ثبت میشود.
\item \textbf{کنشگر اصلی:}\\
مشتری
\item \textbf{کنشگر فرعی:}\\
ندارد
\item \textbf{شرایط اولیه:}\\
مشتری وارد داشبورد شده باشد.
\item \textbf{روند اصلی:}\\
\begin{enumerate}
\item  این مورد کاربرد با درخواست مشتری برای ثبت درخواست جدید شروع میشود.
\item سامانه صفحه درخواست جدید را به مشتری نشان میدهد.
\item سامانه اطلاعات موردنیاز برای ثبت درخواست را به مشتری نمایش میدهد
\item مشتری اطلاعات موردنیاز را وارد میکند(یا به صورت متن یا به صورت فایل)
\item سامانه هزینه ثبت نام را به مشتری نمایش میدهد
\item مشتری درخواست خود را ثبت میکند
\item در صورتی که موجودی مشتری کمتر از هزینه ثبت نام باشد سامانه :
\begin{enumerate}
\item سامانه پیغام خطا میدهد و روند جایگزین ۱ انجام میشود. 
\end{enumerate}

\item در غیراینصورت	:
\begin{enumerate}
\item پیغام ثبت موفق تراکنش به مشتری نشان داده میشود.
\item آدرس مربوط به کارمند مسئول بررسی درخواست به مشتری نمایش داده میشود.
\item مورد کاربردی «بررسی تراکنش» اجرا میشود.
\end{enumerate}

\end{enumerate}

\item \textbf{شرایط پایانی:}\\ 
در قسمت تراکنشهای مربوط به کارمند مسئول، تراکنش ایجاد شده در حالت در انتظار تایید اضافه شود.
\item \textbf{روندهای جایگزین:}\\
روند ۱:\\
\begin{enumerate}
\item پیغام ورود به کیف پول نمایش داده میشود.
\item کاربر ورود به کیف پول را تایید میکند.
\item مورد کاربردی «شارژ کیف پول » اجرا میشود.؟؟
\end{enumerate}

\end{itemize}


\noindent\makebox[\linewidth]{\rule{\paperwidth}{0.4pt}}

\begin{itemize}
\item \textbf{مورد کاربرد:}\\
بررسی تراکنش
\item \textbf{شناسه:}\\
11
\item \textbf{توضیح اجمالی:}\\
کارمند درخواستهای دریافتی از مشتری ها را بررسی میکند.
\item \textbf{کنشگر اصلی:}\\
کارمند
\item \textbf{کنشگر فرعی:}\\
ندارد
\item \textbf{شرایط اولیه:}\\
کارمند وارد داشبورد کارمند شده باشد.
\item \textbf{روند اصلی:}\\
\begin{enumerate}
\item  این مورد کاربرد با درخواست کارمند برای بررسی درخواست جدید شروع میشود.
\item سامانه صفحه بررسی درخواست جدید را به کارمند نشان میدهد.
\item کارمند یکی از درخواست ها را انتخاب میکند.
\item در صورتی که وضعیت درخواست درانتظار تایید نباشد:
\begin{enumerate}
\item در همان صفحه میماند و وارد وضعیت ۳ میشود.
\end{enumerate}
\item در غیر این صورت:
\begin{enumerate}
\item سامانه صفحه مربوط به اطلاعات درخواست مشتری را به کارمند نشان میدهد.
\item کارمند اطلاعات مشتری را بررسی میکند.
\item در صورت صحت اطلاعات:
\begin{enumerate}
\item کارمند وضعیت درخواست را به درحال رسیدگی تغییر میدهد.
\item هزینه تراکنش و کارمزد آن محاسبه میشود
\item هزینه محاسبه شده از حساب ریالی مشتری کم شده و به حساب ریالی شرکت اضافه میشود
\item کارمند پیغام مربوط به نوتیفیکیشن را وارد میکند
\item کارمند ارسال نوتیفیکیشن را تایید میکند
\item مورد کاربردی  «ارسال نوتیفیکیشن کارمند» انجام میشود.
\end{enumerate}
\item در غیر اینصورت:
\begin{enumerate}
\item  در صورتی که کارمند مورد غیر عادی متوجه شود وضعیت تراکنش را غیرعادی میکند.
\item در غیر این صورت وضعیت تراکنش را لغوشده اعلام میکند.
\item کارمند پیغام مربوط به نوتیفیکیشن را وارد میکند
\item کارمند ارسال نوتیفیکیشن را تایید میکند
\item مورد کاربردی  «ارسال نوتیفیکیشن کارمند» انجام میشود.
\end{enumerate}
\end{enumerate}

\end{enumerate}

\item \textbf{شرایط پایانی:}\\ 
در قسمت تراکنشهای مربوط به مشتری، تراکنش درخواستی در وضعیت تایید باشد.\\
مقدار هزینه انجام تراکنش از حساب مشتری کم شده و به حساب ریالی شرکت اضافه شده باشد.\\
\item \textbf{روندهای جایگزین:}\\


\end{itemize}

\noindent\makebox[\linewidth]{\rule{\paperwidth}{0.4pt}}

\begin{itemize}
\item \textbf{مورد کاربرد:}\\
زمانبندی انجام تراکنشها
\item \textbf{شناسه:}\\
12
\item \textbf{توضیح اجمالی:}\\
تمام درخواستها بررسی میشوند و هر درخواستی که از مهلت آن بیش از 24 ساعت گذشته باشد لغو میشود.
\item \textbf{کنشگر اصلی:}\\
زمان
\item \textbf{کنشگر فرعی:}\\
کارمند
\item \textbf{شرایط اولیه:}\\
ندارد
\item \textbf{روند اصلی:}\\
\begin{enumerate}
\item در دوره های زمانی منظم اجرا میشود.
\item سامانه تمام درخواست های ثبت شده را مشاهده میکند
\item هر درخواستی که از زمان شروع آن بیش از 24 ساعت گذشته باشد و جزو درخواستهای تایید نشده باشد به وضعیت لغو در می آید
\item  با توجه به شیوه اطلاع رسانی ثبت شده توسط مشتریهای مربوط به درخواستهای لغو شده پیغام لغو تراکنش برای آنها ارسال میشود.

\end{enumerate}

\item \textbf{شرایط پایانی:}\\ 
وضعیت درخواستهای مشتریانی(و نیز درخواستهای مربوطه در قسمت کارمندان) که بیش از 24 ساعت از آن گذشته و تایید نشده اند به حالت لغو در می آید.\\
\item \textbf{روندهای جایگزین:}\\

\end{itemize}

\noindent\makebox[\linewidth]{\rule{\paperwidth}{0.4pt}}



\end{document}
